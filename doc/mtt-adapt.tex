\section{Boundary-Element Oriented Adaptive Calculation}
\label{adfamu}

In general, when representing the charge in a region by a
multipole expansion, the coefficients $M^m_n$ in (\ref{eq:multi})
are determined from the charge density, $ q( \rho, \alpha, \beta) $, as
\begin{equation}
M^m_n = \int_{region} 
q(\rho, \alpha, \beta) \rho^n Y^{-m}_n(\alpha , \beta) da'.
\label{eq:mulmom}
\end{equation}
In this section, computing the multipole expansion coefficients for
panel charges is examined in more detail, and is shown to lead
naturally to an adaptive multipole algorithm.

\subsection{Computing the Multipole Expansions}

Returning to Figure~(\ref{mulevl}), the potential at the Cartesian
equivalent of $(r_i,\phi_i,\theta_i)$, $(x_i, y_i, z_i)$, due to $d$
distant panels may be approximated with a zeroth-order multipole
expansion, which is equivalent to computing the potential due to a
single charge equal to the sum of the $ d $ panel charges, and located
at the center of the smallest ball enclosing the panels,
\begin{equation}
\psi(x_i,y_i,z_i)\approx\Mb{0}{0}\frac{1}{r_i}.
\label{mulex0}
\end{equation}
Here $(x_i,y_i,z_i)$ is the $i$-th panel's center point relative to the
center of the smallest ball enclosing the distant panels, 
$r_i\trieq\sqrt{x_i^2+y_i^2+z_i^2}$, and
\begin{equation}
\Mb{0}{0}\trieq \sum_{j=1}^{d} q_j.
\end{equation}
The approximation
\begin{equation}
\psi_0(x_i,y_i,z_i)\trieq \Mb{0}{0}\frac{1}{r_i}
\end{equation}
is the zeroth-order multipole expansion for the potential due to the
distant panels.
For accuracy reasons, higher order expansions are typically used.
For example, the first-order multipole expansion for $\psi$ is
\begin{equation}
\psi(x_i,y_i,z_i)\approx\psi_0(x_i,y_i,z_i)+\psi_1(x_i,y_i,z_i),
\end{equation}
with
\begin{equation}
\psi_1(x_i,y_i,z_i)\trieq\Mb{1}{0}\frac{z_i}{r_i^3}-\Mb{1}{1}\frac{x_i}{2r_i^3}-\Mbb{1}{1}\frac{y_i}{2r_i^3},
\label{fstotr}
\end{equation}
where, for panels on which the charge is assumed uniformly distributed,
\begin{eqnarray}
\Mb{1}{0}&\trieq &\sum_{j=1}^{d}\frac{q_j}{a_j}\int_{panel_j}z' da';
\label{moco1}\\
\Mb{1}{1}& \trieq & -2\sum_{j=1}^{d}\frac{q_j}{a_j}\int_{panel_j}x' da';\\
\Mbb{1}{1}& \trieq & -2\sum_{j=1}^{d}\frac{q_j}{a_j}\int_{panel_j}y' da'.
\label{moco2}
\end{eqnarray}
The added potential $\psi_1$
is the field due to a single dipole aligned along the vector
\begin{equation}
(x,y,z) = (\Mb{1}{0}, -\Mb{1}{1}/2, -\Mbb{1}{1}/2).
\end{equation}
In general, the $l$-th order multipole expansion in (\ref{eq:multi}) can
be rewritten in the form
\begin{equation}
\psi(x_i,y_i,z_i)\approx\sum_{n=0}^l\psi_n(x_i,y_i,z_i),
\label{mueval}
\end{equation}
with each $\psi_n$ cooresponding to the potential due to
a $2^n$-pole charge constellation.

\subsection{The Adaptive Algorithm}

The simple multipole algorithm discussed in Section~\ref{mutapp}
uses multipole expansions to represent potentials due to
panel charges inside cubes which
are far enough away from the evaluation point.
The efficiency of this procedure depends on the number of panels in the cubes. 
Consider, for example, a cube containing three panels whose
potential is to be approximated using a first-order multipole
expansion of the form (\ref{mueval}),
\begin{equation}
\psi(x_i,y_i,z_i)\approx\Mb{0}{0}\frac{1}{r_i}+\Mb{1}{0}\frac{z_i}{r_i^3}-\Mb{1}{1}\frac{x_i}{2r_i^3}-\Mbb{1}{1}\frac{y_i}{2r_i^3}.
\label{1oeval}
\end{equation}
The exact potential due to the cube's panels has the form
\begin{equation}
\psi(x_i,y_i,z_i)=P_{i1}q_1+P_{i2}q_2+P_{i3}q_3,
\label{eoeval}
\end{equation}
assuming the three panels are numbered 1, 2 and 3. 
In the capacitance
calculation, the geometry-dependent quantities in (\ref{1oeval}) 
and (\ref{eoeval}) are calculated once and stored for repeated
use in computing the iterates of Algorithm 1.  Thus evaluating
(\ref{1oeval}) involves multiplying four, fixed, geometry-dependent quantities
\begin{equation}
\frac{1}{r_i},\rule{0.2in}{0in}\frac{z_i}{r_i^3},\rule{0.2in}{0in}\frac{x_i}{2r_i^3},\rule{0.2in}{0in}\frac{y_i}{2r_i^3}
\end{equation}
by the charge-dependent multipole coefficients $\Mb{0}{0}$,
$\Mb{1}{0}$, $\Mb{1}{1}$ and $\Mbb{1}{1}$, while (\ref{eoeval})
involves computing only the three products, multiplying the geometric
quantities $P_{ij}$, $j=1$, \ldots, $3$, by the corresponding
charges. Thus evaluating the exact potential from (\ref{eoeval}) requires
one less operation than evaluating the approximate potential using
(\ref{1oeval}), indicating that it will be more efficient to evaluate
the potential due to panels in this cube using $P_{ij}q_j$ products
rather by evaluating multipole expansions.

An adaptive multipole algorithm can be derived from the simplified
approach described in Section~\ref{mutapp} if the potential due to
panel charges in a cube is always evaluated directly, rather than
with a multipole approximation, whenever the number of expansion
coefficients would exceed the number of panels.  A more precise
definition of the computational procedure is given in Algorithm 2
below, which uses some notation which we now introduce.

The cube which contains the entire problem domain is referred to as
the level 0 cube. If the volume of the cube is divided into eight
equally sized child cubes, referred to as level 1 cubes, then each has
the level 0 cube as its parent. The panels are distributed among the
child cubes by associating a panel with a cube if the panel's center
point is contained in the cube. This process can be repeated to
produce $ L $ levels of cubes, and $ L $ partitionings of panels starting
with an $8$-way partitioning and ending with an $ 8^L $-way partitioning.
The number of levels, $ L $, is chosen so that the maximum number of
panels in any finest, or $L$-th, level cube is less than some
threshold (nine is a typical default).  A neighbor of a given cube is
defined as any cube which shares a corner with the given cube or
shares a corner with a cube which shares a corner with the given cube
(note that a cube has a maximum of $ 124 $ neighbors).  Finally, in
the algorithm below it is assumed that for each finest-level cube, a
multipole list has been constructed using a recursive approach similar
to that given in the two-dimensional example above.

\begin{singlespace}
\begin{quote}
\begin{tabbing}
{\bf Algorithm 2:} Adaptive Algorithm for Computing $ p = P q $.\\
T \= \kill
\>T \= \kill
\>\>T \= \kill
\>\>\>T \= \kill
\>\>\>\>T \= \kill
\>\>\>\>\>T \= \kill
\>\>\>\>\>\>T \= \kill
\> Comment: Compute the potential due to nearby charges directly.\\
\>\> {\bf For each} finest-level cube $ i = 1 $ to $ 8^L $ \{ \\
\>\>\> {\bf For each} panel $ j $ in finest-level cube $ i $ \{ \\
\>\>\>\> {\bf Set} $ p_j = 0 $.\\
\>\>\>\> {\bf For each} panel $ k $ in cube $ i $ or its neighbors \{ \\
\>\>\>\>\> {\bf Add} $ P_{jk} q_k $ to $ p_j $.\\
\>\>\>\> {\bf \}}\\
\>\>\> {\bf \}}\\
\>\> {\bf \}}\\
\> Comment: Compute the multipole coefficients from the charge vector $ q $.\\
\> Comment: $order$ is the order of the multipole expansion (typically 2).\\
\>\> {\bf For each} level $ j = L$ to $2$  \{\\
\>\>\> {\bf For each} level $j $ cube $ i = 1 $ to $ 8^j $ \{\\
\>\>\>\> {\bf If} cube $ i $ contains more than $ (order+1)^2 $ panels \{\\
\>\>\>\>\> {\bf Compute} the multipole coefficients for cube $ i $ using \\
\>\>\>\>\> panel charges and/or coefficients of child cube multipole \\
\>\>\>\>\> expansions.\\
\>\>\>\> {\bf \}}\\
\>\>\> {\bf \}}\\
\>\> {\bf \}}\\
\> Comment: Compute the potential due to distant panels.\\
\>\> {\bf For each} finest-level cube $ i = 1 $ to $ 8^L $ \{\\
\>\>\> {\bf For each} cube $j$ in cube $i$'s multipole list \{\\
\>\>\>\> {\bf If} cube $ j $ contains more than $ (order+1)^2$ panels \{\\
\>\>\>\>\> {\bf For each} panel $ k $ in cube $ i $ \{\\
\>\>\>\>\>\> {\bf Evaluate} the multipole expansion for cube $j $ at\\
\>\>\>\>\>\> $(x_k,y_k,z_k)$ and add to $ p_k $.\\
\>\>\>\>\> {\bf \}}\\
\>\>\>\> {\bf \}}\\
\>\>\>\> {\bf Else} \{\\
\>\>\>\>\> {\bf For each} panel $ k $ in cube $ i $ \{\\
\>\>\>\>\>\> {\bf For each} panel $ l $ in cube $ j $ \{\\
\>\>\>\>\>\>\> Add $ P_{kl} q_l $ to $ p_k $.\\
\>\>\>\>\>\> {\bf \}}\\
\>\>\>\>\> {\bf \}}\\
\>\>\>\> {\bf \}}\\
\>\>\> {\bf \}}\\
\>\> {\bf \}}\\
\end{tabbing}
\end{quote}
\end{singlespace}

%%%%%%%%Insert this text just before section 5

%%%%%%%%Jacob

It should be again noted that, as mentioned at the end of
Section~\ref{mutapp}, Algorithm 2 is a simplified version of the
hierarchical multipole algorithm \cite{carrie,greeng88} used in
FASTCAP \cite{nabors91}. The complete algorithm also adaptively
applies local expansions to improve the efficiency of the process of
gathering together multipole expansions.

\subsection{Comparison to Previous Work}

The above approach to making the multipole algorithm adaptive is
specialized to the boundary-element problem.  A more general, but not
as efficient, approach would be to extend to three-dimensions the
two-dimensional adaptive algorithm described in \cite{carrie}.  In
this earlier work, the multipole algorithm is made adaptive by
breaking up the problem domain nonuniformly, in which case lowest level
squares are of different sizes, where the size is chosen so that
each lowest-level square has roughly the same number of panels.

%
% the greegard vs white adaptive alg comparison figure
%
%\begin{figure}
%\centerline{\psfig{figure=/nfs/sobolev/sobolev2/ksn/multi/Doc/MTTfinal/figures/adaptg.eps,width=2.9in}\rule{0.2in}{0in}\psfig{figure=/nfs/sobolev/sobolev2/ksn/multi/Doc/MTTfinal/figures/adaptw.eps,width=2.9in}}
%\caption{Partitionings used in Algorithm~2, (b), and by the adaptive algorithm
%of \protect\cite{carrie}, (a).}
%\label{adaptf}
%\end{figure}
%
%
%

%%&^^
\newcommand{\adaptfcap}{Partitionings used in Algorithm~2, (b), and by the adaptive algorithm of \protect\cite{carrie}, (a).}

\begin{figure}
\setlength{\unitlength}{1.0in}
\begin{picture}(6.5,3.2)
\put(0.25,0.2){
\special{psfile=\figuredir/adaptg.eps,hoffset=-24,voffset=-106,hscale=0.715068,vscale=0.715068}
}
\put(3.35,0.2){
\special{psfile=\figuredir/adaptw.eps,hoffset=-23,voffset=-105,hscale=0.710204,vscale=0.710204}
}
\put(1.65,0){(a)}
\put(4.75,0){(b)}
\end{picture}
\caption{\adaptfcap}
\label{adaptf}
\end{figure}




To see why an adaptive algorithm based on a nonuniform partitioning of
the problem domain can be inefficient, consider computing the
potential at the center of the panel labeled {\bf A}   in
Figure~\ref{adaptf}a.  For this example, a two-dimensional square problem
domain has been recursively quartered, the recursion being halted when
a square had no more than one panel. To compute the potential at the
center of panel  {\bf A} using a multipole algorithm based on this
nonuniform partitioning requires: nine direct potential evaluations, 
for the nine panels in nearest-neighbor squares bordering the square
containing  {\bf A}, and eleven multipole evaluations for the eleven
nonempty squares not bordering  {\bf A}.  Alternatively, if a uniform
partitioning is used, as in Figure~\ref{adaptf}b, then it is easily seen
that only {\it five} multipole evaluations are required.

Such an increase in computational cost can not occur with the adaptive
algorithm given above, as the following theorem states:
\begin{Theorem}
The computational cost of the adaptive approach given in
Algorithm~2 is {\it never greater} than that of the
corresponding nonadaptive algorithm.
\end{Theorem}

The proof follows directly from the fact that the adaptive multipole
algorithm, Algorithm~2, evaluates the potential due to panel charges in a cube
directly, rather than with a multipole approximation, whenever the
number of expansion coefficients would exceed the number of panels.

It should be noted that the overhead cost of maintaining a uniform
domain partitioning has been ignored in this comparison.  If the
distribution of panels is very nonuniform, a matching nonuniform
partitioning, like the approach used in \cite{carrie}, will naturally
generate fewer lowest-level partitions, and hence reduce bookkeepping
overhead. Careful data organization can make this overhead negligible
for typical capacitance calculation problems.  In the examples
presented in Section~\ref{expres}, for which FASTCAP generates
uniform domain partitionings with as many as $ 250,000 $ lowest-level
cubes, the bookkeepping overhead is never significant.
