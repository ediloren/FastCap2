%\documentstyle[11pt]{article}
%\setlength{\topmargin}{-.65in}
%\setlength{\footskip}{0in}
%\setlength{\headheight}{0in}
%\setlength{\headsep}{0in}
%\setlength{\textheight}{10in}
%\setlength{\oddsidemargin}{.4in}
%\setlength{\textwidth}{5.7in}
%\begin{document}

%\input{moments}
%\input{math}

%\section*{Appendix: Multipole Algorithm Formulas}
\section{Multipole Algorithm Formulas}
\label{appfor}

This appendix presents the multipole algorithm expansion, shift
and conversion formulas used in the capacitance extraction algorithm
implementation.
The formulas used are equivalent to those
in the original multipole algorithm formulation
of \cite{gre,fft} but avoid complex arithmetic.
They are obtained by combining complex conjugates in
the original formulas to obtain expressions in the style of \cite{hob}. 
Section~\ref{mopart} defines the real valued
coefficients and the spherical harmonics which together are used to form
the multipole formulas of Section~\ref{muform}.

\subsection{Formula Components}
\label{mopart}

Each multipole or local expansion term involves a 
coefficient multiplying a spherical harmonic. 
When a real coefficient expansion is
used this fact is obscured by the combination of complex conjugates.
However, since the real coefficient expansions are just reorganizations of 
the complex coefficient formulas, the
same coefficients and spherical harmonics appear in slightly different form.

\subsubsection{Real Valued Expansion Coefficients}

Given a multipole or local expansion coefficient, $G_n^m$, the corresponding
real valued coefficients are defined as
\begin{eqnarray}
\Gb{n}{m} &\trieq& \left\{
\begin{array}{ll}
\rule[-5mm]{0mm}{5mm}2\sqrt{\frac{(n+|m|)!}{(n-|m|)!}}\:\mbox{Re}\{G_n^m\}, & |m|>0,\: |m|\leq n;\\
G_n^m, & m=0,\: m\leq n;\\
0, &\mbox{otherwise;}
\end{array}
\right.\label{bardef}\\
\Gbb{n}{m} &\trieq & \left\{
\begin{array}{ll}
\rule[-5mm]{0mm}{5mm}-2\sqrt{\frac{(n+m)!}{(n-m)!}}\:\mbox{Im}\{G_n^m\}, & m>0,\: m\leq n;\\
\rule[-5mm]{0mm}{5mm}2\sqrt{\frac{(n+|m|)!}{(n-|m|)!}}\:\mbox{Im}\{G_n^m\}, & m<0,\: |m|\leq n;\\
0, & \mbox{otherwise.}
\end{array}
\right.\label{dbrdef}
\end{eqnarray}
All the multipole algorithm formulas are converted to real coefficients using
these substitutions for the complex coefficients.

\subsubsection{Spherical Harmonics}

The functions
\begin{equation}
Y_n^m(\theta, \phi)\trieq\sqrt{\frac{(n-|m|)!}{(n+|m|)!}}P_n^{|m|}(\cos\theta)e^{im\phi},
\label{sphhar}
\end{equation}
are called surface spherical harmonics.  
A surface spherical harmonic
is part of a solution to Laplace's equation obtained by separation
of variables. 
Here, as in \cite{jac}, 
a surface spherical harmonic is the product of the
elevation ($\theta$) and azimuth ($\phi$) components of the solution.
Unlike the usual definition, however,
a normalization constant is omitted following \cite{gre,fft}.
The complete solution is a product of $Y_n^m$ and a power of $r$, 
the radial coordinate. The result, for example
\begin{equation}
\frac{1}{r^{n+1}}Y_n^m(\theta, \phi),
\label{spfrac}
\end{equation}
is called a spherical harmonic.


The function $P_n^{m}(\cos\theta)$ is the associated Legendre function 
of the first kind with degree $n$ and order $m$. These functions are
defined only when $n$ is a non-negative integer and $m$ is an integer
such that $-n\leq m\leq n$. For convenience any $P_n^{m}(\cos\theta)$
whose indices do not satisfy these restrictions is taken to be zero.

%To apply the multipole formulas to order $l$ expansions,
%all the possible associated Legendre
%functions with $0\leq n\leq l$ or $2l$, 
%depending on the formula, are required. For example an order $l=3$
%multipole expansion shift requires the Legendre function evaluations
%\[
%\begin{array}{llll}
%P_0^0(\cos\alpha)&&&\\
%P_1^0(\cos\alpha)&P_1^1(\cos\alpha)&&\\
%P_2^0(\cos\alpha)&P_2^1(\cos\alpha)&P_2^2(\cos\alpha)&\\
%P_3^0(\cos\alpha)&P_3^1(\cos\alpha)&P_3^2(\cos\alpha)&P_3^3(\cos\alpha)\\
%\end{array}
%\]
The recursion
\begin{eqnarray}
\lefteqn{(n-m) P_n^m(\cos\alpha)}\nonumber\\
&&=(2n-1)\: \cos\alpha\: P_{n-1}^m(\cos\alpha) - (n+m-1)P_{n-2}^m(\cos\alpha),
\label{nmrecl}
\end{eqnarray}
valid for $0\leq m\leq n-2$,
and the formulas
\begin{eqnarray}
P_m^m(\cos\alpha)=\frac{(2m)!}{2^m m!} (-\sin\alpha)^m,&&0\leq m,\label{mmlegn}\\
P_{m+1}^{m}(\cos\alpha)=(2m+1) \cos\alpha P_m^m(\cos\alpha),&&0\leq m,\label{mm1leg}
\end{eqnarray}
can be used to recursively evaluate the Legendre functions \cite{hob,mag}.
%Formula (\ref{nmrecl})
%is one standard recursion on $n$ and (\ref{mm1leg}) is a special case
%of another standard recursion on $n$ \cite{hob, mag}. 
%%\cite[pg.\ 54]{mag}. 
%The expression (\ref{mmlegn}) follows
%from the  definition 
%\begin{equation}
%P_n^m(\cos\alpha) \trieq (-\sin\alpha)^m \frac{d^m}{d(\cos\alpha)^m} P_n^0(\cos\alpha)
%\end{equation}
%and the form of $P_n^0(\cos\alpha)$. The table of evaluations is generated
%one column at a time using the recursion (\ref{nmrecl}) with starting
%values from (\ref{mmlegn}) and (\ref{mm1leg}). The standard recursion
%on $m$ is avoided since it often produces errors due to 
%cancellation \cite{low}.
%%\cite[pg.\ xvii]{low}.

\subsection{Real Coefficient Multipole Algorithm Formulas}
\label{muform}

Using the real valued coefficients and the spherical harmonics of the previous
section, the multipole algorithm formulas used in the capacitance extraction
algorithm are obtained. The resulting real coefficient formulas eliminate the
need for complex arithmetic and square root calculations.

\subsubsection{Multipole Expansion (Q2M, M2P)}

The order $l$ multipole expansion approximation to the potential, $\psi$,
at the point
$(r, \theta, \phi)$ 
is
\begin{equation}
\psi(r, \theta, \phi)\approx \sum_{n=0}^{l}\sum_{m=-n}^{n}\frac{M_n^m}{r^{n+1}}Y_n^m(\theta, \phi).
\end{equation}
Applying the definition of the surface spherical harmonic,
$Y_n^m$,
gives
\begin{equation}
\psi(r, \theta, \phi)\approx\sum_{n=0}^{l}\sum_{m=-n}^{n}\frac{M_n^m}{r^{n+1}}\sqrt{\frac{(n-|m|)!}{(n+|m|)!}}P_n^{|m|}(\cos\theta)e^{im\phi}.
\end{equation}
Substituting the real coefficients using (\ref{bardef}) and (\ref{dbrdef}) 
yields
the real coefficient multipole expansion,
\begin{equation}
\psi(r, \theta, \phi) \approx \sum_{n=0}^{l}\frac{1}{r^{n+1}}\sum_{m=0}^n
\frac{(n-m)!}{(n+m)!}P_n^m(\cos\theta)
\left[\Mb{n}{m}\cos(m\phi)+\Mbb{n}{m}\sin(m\phi)\right].
\label{remult}
\end{equation}
The complex coefficient local expansion conversion is nearly identical.

A multipole expansion is constructed from
$k$ charges with strengths $q_i$ and positions
$(\rho_i, \alpha_i, \beta_i)$, $i=1$, \ldots, $k$, using
\begin{equation}
M_n^m \trieq \sum_{i=1}^{k}\, q_i\, \rho_i^n\, Y_n^{-m}(\alpha_i, \beta_i).
\end{equation}
Substituting (\ref{bardef}) and (\ref{dbrdef}) gives expressions for
the real multipole coefficients corresponding to a set of $k$ charges,
\begin{eqnarray}
\Mb{n}{m} &=& \left\{
\begin{array}{ll}
\rule[-5mm]{0mm}{5mm}2\sum_{i=1}^{k}\, q_i\, \rho_i^n\, P_n^{|m|}(\cos\alpha_i)\, \cos(m\beta_i), & |m|>0,\: |m|\leq n;\\
\rule[-5mm]{0mm}{5mm}\sum_{i=1}^{k}\, q_i\, \rho_i^n\, P_n^{0}(\cos\alpha_i), & m=0,\: m\leq n;\\
0, &\mbox{otherwise;}
\end{array}
\right.\label{barmul}\\
\Mbb{n}{m} &=& \left\{
\begin{array}{ll}
\rule[-5mm]{0mm}{5mm}2\sum_{i=1}^{k}\, q_i\, \rho_i^n\, P_n^{|m|}(\cos\alpha_i)\, \sin(m\beta_i), & |m|>0,\: |m|\leq n;\\
0, & \mbox{otherwise.}
\end{array}
\right.\label{dbrmul}
\end{eqnarray}


\subsubsection{Local Expansion (Q2L, L2P)}

The order $l$ local expansion approximation to the potential, $\psi$,
at the point
$(r, \theta, \phi)$ 
is
\begin{equation}
\psi(r, \theta, \phi) \approx \sum_{n=0}^{l}\sum_{m=-n}^{n}L_n^mY_n^m(\theta, \phi)r^n,
\end{equation}
or
\begin{equation}
\psi(r, \theta, \phi) \approx\sum_{n=0}^{l}\sum_{m=-n}^{n}L_n^m\sqrt{\frac{(n-|m|)!}{(n+|m|)!}}P_n^{|m|}(\cos\theta)e^{im\phi}r^n.
\end{equation}
Substituting the real coefficients using (\ref{bardef}) and (\ref{dbrdef}) 
yields
the real coefficient local expansion,
\begin{equation}
\psi(r, \theta, \phi) \approx
\sum_{n=0}^{l}r^n\sum_{m=0}^n\frac{(n-m)!}{(n+m)!}
P_n^m(\cos\theta)\left[\Lb{n}{m}\cos(m\phi)+\Lbb{n}{m}\sin(m\phi)\right].
\label{relocl}
\end{equation}

A local expansion is constructed from
$k$ charges with strengths $q_i$ and positions
$(\rho_i, \alpha_i, \beta_i)$, $i=1$, \ldots, $k$, using
\begin{equation}
L_n^m \trieq \sum_{i=1}^{k}\, \frac{q_i}{\rho_i^{n+1}}\, Y_n^{-m}(\alpha_i, \beta_i).
\end{equation}
Substituting (\ref{bardef}) and (\ref{dbrdef}) gives expressions for
the real multipole coefficients corresponding to a set of charges,
\begin{eqnarray}
\Lb{n}{m} &=& \left\{
\begin{array}{ll}
\rule[-5mm]{0mm}{5mm}2\sum_{i=1}^{k}\, \frac{q_i}{\rho_i^{n+1}}\, P_n^{|m|}(\cos\alpha_i)\, \cos(m\beta_i), & |m|>0,\: |m|\leq n;\\
\rule[-5mm]{0mm}{5mm}\sum_{i=1}^{k}\, \frac{q_i}{\rho_i^{n+1}}\, P_n^{0}(\cos\alpha_i), & m=0,\: m\leq n;\\
0, &\mbox{otherwise;}
\end{array}
\right.\label{barloc}\\
\Lbb{n}{m} &=& \left\{
\begin{array}{ll}
\rule[-5mm]{0mm}{5mm}2\sum_{i=1}^{k}\, \frac{q_i}{\rho_i^{n+1}}\, P_n^{|m|}(\cos\alpha_i)\, \sin(m\beta_i), & |m|>0,\: |m|\leq n;\\
0, & \mbox{otherwise.}
\end{array}
\right.\label{dbrloc}
\end{eqnarray}

\subsubsection{Multipole Expansion Shift (M2M)}

Consider a multipole expansion about the point
$(\rho, \alpha, \beta)$. The potential at a given point results when its
coordinates relative to $(\rho, \alpha, \beta)$ are substituted into
the expansion. If the expansion about $(\rho, \alpha, \beta)$
has coefficients $O_n^m$, then the coefficients of a shifted multipole
expansion about the origin, $N_j^k$, are given by
\begin{equation}
N_j^k = \sum_{n=0}^j\sum_{m=-n}^n\frac{\sqrt{(j+k)!(j-k)!}\:i^{|k|-|m|-|k-m|}\:Y_n^{-m}(\alpha, \beta)\:O_{j-n}^{k-m}\rho^n}{\sqrt{(j-n+k-m)!\:(j-n-k+m)!\:(n+m)!\:(n-m)!}}.
\end{equation}
Substituting for the surface
spherical harmonics using (\ref{sphhar}) and for the
complex coefficients with (\ref{bardef}) and (\ref{dbrdef}) gives the real
coefficient multipole expansion shift formulas for
$j\geq k\geq 0$, 
\begin{eqnarray}
\lefteqn{\Nb{j}{k}=
(j+k)!\sum_{n=0}^j\rho^n\sum_{m=0}^nf_{M}(m, k)
\frac{P_n^{m}(\cos\alpha)}{(n+m)!}}\nonumber\\
&&\cdot\left\{\frac{i^{k-m-|k-m|}}{(j-n+|k-m|)!}
\left[\Ob{j-n}{k-m}\cos(m\beta)-\Obb{j-n}{k-m}\sin(m\beta)\right]\right.
\nonumber\\
&&\left.+\frac{(-1)^m}{(j-n+k+m)!}
\left[\Ob{j-n}{k+m}\cos(m\beta)+\Obb{j-n}{k+m}\sin(m\beta)\right]\right\};
\\
\lefteqn{\Nbb{j}{k}=
(j+k)!\sum_{n=0}^j\rho^n\sum_{m=0}^nf_{M}(m, k)
\frac{P_n^{m}(\cos\alpha)}{(n+m)!}}\nonumber\\
&&\cdot\left\{\frac{i^{k-m-|k-m|}}{(j-n+|k-m|)!}
\left[\Ob{j-n}{k-m}\sin(m\beta)+\Obb{j-n}{k-m}\cos(m\beta)\right]\right.
\nonumber\\
&&\left.+\frac{(-1)^m}{(j-n+k+m)!}
\left[-\Ob{j-n}{k+m}\sin(m\beta)+\Obb{j-n}{k+m}\cos(m\beta)\right]\right\}.
\end{eqnarray}
Here
\begin{equation}
f_{M}(m, k)\trieq \left\{
\begin{array}{ll}
1,&m\neq 0,\: k\neq 0;\\
1/2,&m=0,\: k\neq 0;\\
1/2,&m\neq 0,\: k=0;\\
1/2,&m=0,\: k=0.
\end{array}\right.
\end{equation}
%Since $\Nb{j}{-k}=\Nb{j}{k}$ and $\Nbb{j}{-k}=-\Nbb{j}{k}$,
%the real coefficients with negative superscripts are never explicitly calculated.
%Furthermore no coefficients with negative subscripts are required in any of the
%expansions and all of the coefficients with $|k|>j$ are zero by
%definition. Thus the coefficients with  $j\geq k\geq 0$ form a complete
%multipole expansion set. The same reasoning
%applies to the coefficients calculated by the other shift and conversion
%operations.

\subsubsection{Multipole to Local Expansion Conversion (M2L)}

An order $l$ multipole expansion about the point
$(\rho, \alpha, \beta)$,
with coefficients $O_n^m$, can be converted to an order $l$ local
expansion about the origin, with coefficients $N_j^k$, using
\begin{equation}
N_j^k = \sum_{n=0}^l\sum_{m=-n}^n\frac{\sqrt{(j+n+m-k)!\:(j+n-m+k)!}\:i^{|k-m|}\:Y_{j+n}^{m-k}(\alpha, \beta)\:O_n^m}{\sqrt{(n+m)!\:(n-m)!\:(j+k)!\:(j-k)!}\:(-1)^n\:i^{|k|+|m|}\:\rho^{j+n+1}}.
\end{equation}
Substituting for the surface
spherical harmonics using (\ref{sphhar}) and for the
complex coefficients with (\ref{bardef}) and (\ref{dbrdef}) gives the real
coefficient multipole to local expansion conversion formulas for
$l\geq j\geq k\geq 0$,
\begin{eqnarray}
\lefteqn{\Nb{j}{k}=
\frac{f_{L}(k)}{\rho^j(j-k)!}
\sum_{n=0}^l\frac{(-1)^n}{\rho^{n+1}}\sum_{m=0}^n}\nonumber\\
&&P_{j+n}^{|m-k|}(\cos\alpha)\frac{(j+n-|m-k|)!}{(n+m)!}\:i^{|k-m|-k-m}
\nonumber\\
&&\rule{0.25in}{0in}\cdot\left\{\Ob{n}{m}\cos[(m-k)\beta]+\Obb{n}{m}\sin[(m-k)\beta]\right\}\nonumber\\
&&+P_{j+n}^{m+k}(\cos\alpha)\frac{(j+n-m-k)!}{(n+m)!}\nonumber\\
&&\rule{0.25in}{0in}\cdot\left\{\Ob{n}{m}\cos[(m+k)\beta]+\Obb{n}{m}\sin[(m+k)\beta]\right\};
\\
\lefteqn{\Nbb{j}{k}=
\frac{1}{\rho^j(j-k)!}
\sum_{n=0}^l\frac{(-1)^n}{\rho^{n+1}}\sum_{m=0}^n}\nonumber \\
&&P_{j+n}^{|m-k|}(\cos\alpha)\:\frac{(j+n-|m-k|)!}{(n+m)!}\:i^{|k-m|-k-m}\nonumber\\
&&\rule{0.25in}{0in}\cdot\left\{-\Ob{n}{m}\sin[(m-k)\beta]+\Obb{n}{m}\cos[(m-k)\beta]\right\}\nonumber\\
&&+P_{j+n}^{m+k}(\cos\alpha)\frac{(j+n-m-k)!}{(n+m)!}\nonumber\\
&&\rule{0.25in}{0in}\cdot\left\{\Ob{n}{m}\sin[(m+k)\beta]-\Obb{n}{m}\cos[(m+k)\beta]\right\}.
\end{eqnarray}
Here
\begin{equation}
f_{L}(k) = \left\{
\begin{array}{ll}
1,&k\neq 0;\\
1/2,&k=0.
\end{array}\right.
\label{fm2ldef}
\end{equation}

\subsubsection{Local  Expansion Shift (L2L)}

An order $l$ local expansion about the point
$(\rho, \alpha, \beta)$,
with coefficients $O_n^m$, can be converted to an order $l$ local
expansion about the origin, with coefficients $N_j^k$, using
\begin{equation}
N_j^k = \sum_{n=j}^l\sum_{m=-n}^n\frac{\sqrt{(n+m)!(n-m)!}\:i^{|m|-|k|-|m-k|}\:Y_{n-j}^{m-k}(\alpha, \beta)\:O_n^m\:\rho^{n-j}}{\sqrt{(n-j+m-k)!(n-j-m+k)!(j+k)!(j-k)!}\:(-1)^{n-j}}.
\end{equation}
Substituting for the surface
spherical harmonics using (\ref{sphhar}) and for the
complex coefficients with (\ref{bardef}) and (\ref{dbrdef}) gives the real
coefficient local expansion shift formulas
for $j\geq k\geq 0$,
\begin{eqnarray}
\lefteqn{\Nb{j}{k}=
\frac{f_{L}(k)}{(-\rho)^j(j-k)!}
\sum_{n=j}^l(-\rho)^n\sum_{m=0}^n}\nonumber\\
&&P_{n-j}^{|m-k|}(\cos\alpha)
\frac{i^{m-k-|m-k|}}{(n-j+|m-k|)!(n-m)!}\nonumber\\
&&\rule{0.25in}{0in}\cdot\left\{\Ob{n}{m}\cos[(m-k)\beta]+\Obb{n}{m}\sin[(m-k)\beta]\right\}\nonumber\\
&&+P_{n-j}^{m+k}(\cos\alpha)\frac{(-1)^k}{(n-j+m+k)!(n-m)!}\nonumber\\
&&\rule{0.25in}{0in}\cdot\left\{\Ob{n}{m}\cos[(m+k)\beta]+\Obb{n}{m}\sin[(m+k)\beta]\right\};
\\
\lefteqn{\Nbb{j}{k}=
\frac{1}{(-\rho)^j(j-k)!}
\sum_{n=j}^l(-\rho)^n\sum_{m=0}^n}\nonumber\\
&&P_{n-j}^{|m-k|}(\cos\alpha)
\frac{i^{m-k-|m-k|}}{(n-j+|m-k|)!(n-m)!}\nonumber\\
&&\rule{0.25in}{0in}\cdot\left\{-\Ob{n}{m}\sin[(m-k)\beta]+\Obb{n}{m}\cos[(m-k)\beta]\right\}\nonumber\\
&&+P_{n-j}^{m+k}(\cos\alpha)\frac{(-1)^k}{(n-j+m+k)!(n-m)!}\nonumber\\
&&\rule{0.25in}{0in}\cdot\left\{\Ob{n}{m}\sin[(m+k)\beta]-\Obb{n}{m}\cos[(m+k)\beta]\right\}.
\end{eqnarray}
The function $f_{L}(k)$ is given by (\ref{fm2ldef}).

%\end{document}
