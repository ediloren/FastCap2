%
% the parallelepiped and its reference points
%
% for fig2ug.awk
%%%&^ pipeeg.ps
\begin{figure}
\centerline{
\psfig{figure=\figuredir/pipeeg.ps,height=3.0in}
}
\begin{picture}(6.25,0)
\thicklines
\put(1.3,1.8){\makebox(1.0,0.2)[r]{{\tt c1} = (1 0 1)}}
\put(4,1.8){{\tt c2} = (0 1 1)}
\put(4.2,1.4){\vector(-1,0){1.0}}
\put(4.3,1.35){{\tt cr} = (1 1 1)}
\put(3.1,0.4){{\tt c3} = (1 1 0)}
\put(3.1,1.8){\tt t}
\put(2.7,1.25){\tt fl}
\put(3.45,1.25){\tt fr}
\end{picture}
\caption{The discretization generated by the command 
{\tt pipedgen -cr 1 1 1  -c1 1 0 1  -c2 0 1 1  -c3 1 1 0  -d}, showing
how the faces are identified.  The axes are two units long.}
\label{pipeeg}
\end{figure}
