%
% should be an appendix
%
\subsection{Bugs}

This section attempts to itemize the major shortcomings of {\tt fastcap}
and the input file generator programs. 
The user is encouraged to report any other bugs by sending mail
to \verb~fastcap-bug@rle-vlsi.mit.edu~.

\subsubsection*{{\tt fastcap}}

\begin{enumerate}
\item List file provides for only translations, not rotations.
\item Translations of panels input via {\tt stdin} or {\tt input file}
are not possible.  Use a list file when translations are required.
\item All input dimensions are assumed in meters. There is currently
no way to scale the problem on the command line. The capacitance
of the true geometry is equal to the calculated capacitance multiplied
by the factor required to convert the input length units (always meters)
to the desired units.
\item Dielectric interfaces can only be input with a list file
(only conductors can be input through {\tt stdin} and {\tt input file}).
Use a list file for all dielectric interfaces.
\item The  dielectric surrounding all conductors
input through {\tt stdin} and {\tt input file} always has the 
permittivity of free space
multiplied by the permittivity factor specified with {\tt -p} (1.0 by
default).
Use a list file if other values are required.
\item Dielectric interfaces have no names and therefore cannot be
removed selectively from the input or charge distribution plots
with command line options.
To remove dielectric interfaces selectively from the input, comment out
lines in the list file.  To remove all dielectric interfaces from
charge plots use {\tt -rd}.
\item Infinitesimally thin conductors on dielectric interfaces (included
using lines starting with `{\tt B}' in list files), are not supported
in this release.
\item All surfaces are listed in the {\tt Input surfaces} part of
the output even if all the conductors in some of them have been
removed using {\tt -ri}.
\item The PATRAN neutral file interface has very little error checking.
If {\tt fastcap} is given an incorrect format neutral file it either
hangs or dumps core.
\item The actual postscript filenames used by {\tt fastcap -m} or 
{\tt fastcap -q<cond list>} are derived from the input file names and cannot
be directly changed by the user.
\item Postscript files produced with {\tt fastcap -m} or 
{\tt fastcap -q<cond list>} produce panel ordering errors (and
usually print warning messages) when panels intersect. However, such
discretizations are illegal.
\item Occasionally the postscript pictures are incorrect even
for legal discretizations. There are two known bugs leading
to this problem.  The first is often avoided by changing
the view angle slightly. The second is caused by problems with
dimensions on the order of $10^{-4}$ or less. This problem can
only be avoided by rescaling the input. 
%
% This error is due to two faces intersecting only at corners; ie
% the line(s) of one pass(passes) through the corner(s) of the other
% - no easy fix; right now all corner hits are not considered intersections
% - very strong function of view angle, however, so can avoid (sa comp book)
\item The panel sorting algorithm used to write out the postscript files
takes time proportional to the number of panels squared.  This limits
its usefulness to problems with less than a few thousand panels. For
larger problems the paneling should be done hierarchically so that
a coarser discretization can be used for pictures. For problems built
using the generic input file generators, the {\tt -d} option provides
this function.
\item When axes are included in postscript files using the
{\tt -x<axes length>} option, the axes' two-dimensional projections
appear in the postscript file before the panel fills. This means
that the object should be between the view point and the axes, otherwise
the axes can be obscured strangely. 
Thus
the option works best when viewing objects that lie entirely in
the positive orthant from a view point in the positive orthant.
\item The bounding box in postscript files produced with {\tt -q} is
incorrect if the shading key is included. The bounding box printed
is that of the shaded image without the key.

\end{enumerate}

\subsubsection*{\tt cubegen}

All the functions of the cube input file generator {\tt cubegen}
can be performed by {\tt pipedgen}, although {\tt cubegen}'s interface
is often more convenient.

\subsubsection*{\tt pyragen}

The pyramid input file generator {\tt pyragen} is currently
only capable of the default discretization, with three panels on a
side (also results from {\tt -n3}), and no discretization (using {\tt -d}).

\subsubsection*{\tt capgen}

All the functions of the 
parallel plate capacitor input file generator {\tt capgen}
can be performed by both {\tt cubegen} and {\tt pipedgen}, unless more
than two plates are required.

