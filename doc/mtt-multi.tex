\section{The Hierarchical Multipole Algorithm}
\label{mutapp}

A complete description of the hierarchical multipole
algorithm is not given here; the original description is 
in \cite{greeng87,greeng88}, and
its application to capacitance extraction is described in
\cite{nabors91}.  Instead we describe the expansion approximation
and examine a simplified two-dimensional example which both exhibits
the method's salient features, and motivates the adaptive algorithm
and the preconditioner described in subsequent sections.

\subsection{Multipole Expansions}

%A standard approach for approximating the far-field due to a confined
%charge distribution is to represent the charge with multipole
%expansions\cite{jackson}.
Multipole expansions are often used to approximate the far field due to
a confined charge distribution \cite{jackso75}. 
For example, consider evaluating the
potential $p_i$ at the center of a panel $i$, 
$(r_i, \phi_i , \theta_i )$, due to
a collection of $ d $ distant panels, as in Figure~\ref{mulevl}.  The
potential due to the surface charges on those $ d $ panels is given
approximately by the truncated multipole expansion
\begin{equation}
\psi(r_i, \phi_i , \theta_i ) \approx \sum_{n=0}^{l} \sum_{m=-n}^{n}
\frac{M^m_n}{r_i^{n+1}} Y^m_n(\phi_i , \theta_i )
\label{eq:multi}
\end{equation}
where the spherical coordinates
of the evaluation location are measured relative to
the origin of the multipole expansion, $ Y^m_n (\phi_i , \theta_i ) $ 
are the surface spherical harmonics, $ M^m_n $ are the multipole coefficents
determined from the panel charges, and $ l $ is the expansion order.

Given the multipole coefficients, the same multipole expansion can be
used to quickly, but approximately, evaluate the potential at many
panel centers. For example, in Figure~\ref{mulevl}, there are $ d $
charged panels, and $d$ panel centers where the potential must be
evaluated. A direct calculation of those potentials requires order
$d^2$ operations, but only order $d$ operations are needed if the
multipole expansion is used (assuming the expansion order $ l $ is fixed).

In the Figure~\ref{mulevl} case,
the error due to truncating the
multipole expansion is bounded \cite{greeng88}, as in 
\begin{equation}
\left| \psi(r_i, \phi_i , \theta_i ) - 
\sum_{n=0}^{l} \sum_{m=-n}^{n}
\frac{M^m_n}{r_i^{n+1}} Y^m_n(\phi_i , \theta_i ) \right| \leq K_1
\left( \frac{R}{r_i} \right)^{l+1} \leq K_1 \left( \frac{R}{r} \right)^{l+1}.
\label{mulerr}
\end{equation}
The quantities $r$ and $R$ are as in Figure~\ref{mulevl} and $K_1$ is
a constant independent of the multipole expansion order, $l$. The
bound indicates that the multipole potential evaluations converge
more rapidly with expansion order as the minimum distance between the
panel charges and the evaluation points increases.

%%&^^
\newcommand{\mulevlcap}{The evaluation of the potential at $(r_i,\phi_i,\theta_i)$.}

\begin{figure}
\setlength{\unitlength}{1.0in}
\begin{picture}(6.5,2.7)
\put(0.25,0){
\special{psfile=\figuredir/mulevl.ps,hoffset=-40,voffset=-450,hscale=0.933045,vscale=0.933045}
}
\thicklines
\put(0.25,2.4){$(r_i,\phi_i,\theta_i)$}
\put(1.65,1.07){$r$}
\put(3.52,1.75){$R$}
\put(2.2,1.9){$r_i$}
\put(3.2,2.4){$d$ panels}
\put(3.6,2.35){\vector(0,-1){0.2}}
\put(2.7,2.8){$d$ evaluation points}
\put(2.65,2.85){\vector(-3,-1){0.5}}
\put(4.1,2.85){\vector(3,-1){0.5}}
\put(2.3,0.23){$=$ multipole expansion}
\end{picture}
\caption{\mulevlcap}
\label{mulevl}
\end{figure}

In order to ensure that the error bound in (\ref{mulerr}) tightens
sufficiently with each increase in expansion order $l$, the
hierarchical multipole algorithm uses a
multipole expansion  to represent the effect of charge in a
region only if the radius of the region, $ R $, is less than half the
distance between the region's center and the evaluation point, denoted
$ r $. For example, in Figure~\ref{dblevl} two groups of panels
are represented by a multipole expansions of order $l$, and by the
above criteria, both can be used to evaluate the potential at panel
$i$'s center, as $R/r=3R/3r<0.5$.
%%&^^
\newcommand{\dblevlcap}{The evaluation of two multipole expansions with the same error bound.}

\begin{figure}
\setlength{\unitlength}{1.0in}
\begin{picture}(6.5,2.7)
\put(1.75,0){
\special{psfile=\figuredir/dblevl.ps,hoffset=-186,voffset=-230,hscale=0.947368,vscale=0.947368}
}
\put(1.73,1.0){$(r_i,\phi_i,\theta_i)$}
\put(2.0,1.48){$r$}
\put(2.35,1.95){$R$}
\put(2.55,1.35){$3r$}
\put(3.6,2.0){$3R$}
\put(2.62,0.1){$=$ multipole expansion}
\end{picture}
\caption{\dblevlcap}
\label{dblevl}
\end{figure}


\subsection{A Two-Dimensional Example}

The aggregation of distant tiles into multipole expansions which can be
used to evaluate potentials at many panel centers is the source of the
hierarchical multipole algorithm's efficiency.  Maintaining this
efficiency for general distributions of panels while controlling error
is ensured by exploiting a hierarchical partitioning of the problem
domain, the smallest cube containing all the conductors.
%%&^^
\newcommand{\direvalcap}{The evaluation of the potential at 
$(r_i,\phi_i,\theta_i)$.}

\begin{figure}
\setlength{\unitlength}{1.0in}
\begin{picture}(6.5,6.5)
\put(0,0.25){
\special{psfile=\figuredir/direval.eps,hoffset=-25,voffset=-111,hscale=0.747405,vscale=0.747405}
}
\put(0.1176,4.926){$(r_i,\phi_i,\theta_i)$}
\put(0.1176,1.197){$(r_i,\phi_i,\theta_i)$}
\put(3.4941,4.926){$(r_i,\phi_i,\theta_i)$}
\put(3.4941,1.197){$(r_i,\phi_i,\theta_i)$}
\put(1.9618,1.0765){$R$}
\put(1.235,1.4794){$r$}
\put(3.8588,6.3294){square 1}
\put(5.371,6.3294){square 2}
\put(5.371,4.9176){square 3}
%\put(3.4588,4.5176){4}
\put(3.529,5.915){$R_1$}
\put(4.2494,5.7941){$r_1$}
\put(5.2882,6.0647){$r_2$}
\put(5.7412,5.9147){$R_2$}
\put(5.2176,4.8559){$r_3$}
\put(5.7412,4.2824){$R_3$}
\put(1.48,3.7){(a)}
\put(4.83,3.7){(b)}
\put(1.48,0){(c)}
\put(4.83,0){(d)}
\end{picture}
\caption{\direvalcap}
\label{direval}
\end{figure}


Consider, for example, evaluating the potential at some point
$(r_i,\phi_i,\theta_i)$ in Figure~\ref{direval} due to panel charges
inside the illustrated problem domain. A first partitioning would be
to break the problem into four smaller squares, leaving
$(r_i,\phi_i,\theta_i)$ somewhere in the lower left square
(Figure~\ref{direval}b)\footnote{ In the three-dimensional problem,
the equivalent partitioning would be to divide a cube into eight
smaller cubes.}.  To ensure that the errors due to truncating the
multipole expansion shrink rapidly with increasing expansion order, multipole
expansions will not be used to represent the charges in squares 1, 2
and 3, when evaluating the potential at points in the lower-left
square, because $R_1/r_{1}$, $R_2/r_{2}$ and $R_3/r_{3}$ in
Figure~\ref{direval}b are all greater than $ 0.5 $.  For the
particular example evaluation point in the lower-left square, the
charge in square 2 is distant enough to satisfy the criteria for using
multipole expansions.  However, a more detailed study of the
hierarchical multipole algorithm than we will consider here would show
that it is not efficient to exploit such special cases.

Squares 1, 2 and 3 are each divided into four squares, as in
Figure~\ref{direval}c, to produce smaller regions which can
possibly satisfy the criteria for representation by a multipole
expansion. In fact, many of the smaller squares do satisfy the
criteria, as can be seen by examining the illustrated case, for which
$R/r $ is less than $0.5$.  Thus, at the end of this
partitioning step, all the charges in 
the squares marked with an M in Figure~\ref{direval}c will 
be represented with a 
multipole expansion when evaluating
the potential at points in the square containing $(r_i,\phi_i,\theta_i)$.

In order for the multipole expansions to be used to represent the potential
due to panel charges contained in the unmarked squares of
Figure~\ref{direval}c, these squares are partitioned further, as in
Figure~\ref{direval}d.  Then, as before, the distance criteria implies that
multipole expansions can be used to represent the panel
charges in all but a few squares near the square containing the
evaluation point.  If it is determined not to partition any further
than is indicated in Figure~\ref{direval}d, the potential $p_i$, at
$(r_i,\phi_i,\theta_i)$, can then be computed by summing a ``near'' or
direct term and a ``far'' or multipole term. That is, the ``near''
contribution to $ p_i $ is due to panel charges in the nine unmarked
squares in Figure~\ref{direval}d, and is computed directly from
$P_{ij}q_j$ products.  The ``far'' contribution to $ p_i $ is due to
distant panels charges and is determined by evaluating the $ 25 $
mulitpole expansions indicated in Figure~\ref{direval}d.  In the next
section, we will refer to the list of squares associated with those
$25 $ multipole expansions as the {\it multipole list } for the square
containing $(r_i,\phi_i,\theta_i)$.

In general, the number of partitioning levels, $ L $, for a given
problem domain is selected so that the squares on the finest level
each have no more than $ k $ panels (typically $ k$ is of the order of
ten).  Then for a uniform distribution of panels, the number of
partitioning levels will be given by $ L = \log{\frac{n}{k}} $.  Since
the number of multipole expansions on each partitioning level which
contribute to $ p_i $ is bounded by a constant, each potential
evaluation involves order $\log{n}$ multipole expansion evaluations.
Also, since each lowest level square has no more than $ k $ panels,
the direct contribution to $ p_i $ is bounded by a constant.
Therefore, as evaluating the entire potential vector requires $n$
evaluations of this type, the above multipole approach is an order
$n\log{n}$ algorithm for computing an approximation 
to $Pq$.\footnote{The analysis is similar
for the three-dimensional case. The primary difference is that in the
three-dimensional case space is partitioned into cubes, and when
cubes are subpartitioned, they generate eight smaller cubes.} The
hierarchical multipole algorithm given in~\cite{greeng88}, and used in
the FASTCAP program described below, is more sophisticated than the
above approach suggests. In particular, multipole evaluations are
efficiently combined into local expansions in such a way as to reduce
the number of operations to order $n$. However, for purposes of
describing the adaptive algorithm and the preconditioning techniques
below, the simplified algorithm above is sufficiently detailed.








