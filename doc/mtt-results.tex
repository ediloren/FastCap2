\section{Experimental Results}
\label{expres}

In this section, results from computational experiments are presented
to demonstrate the efficiency and accuracy of the
preconditioned, adaptive, multipole-accelerated (PAMA) 3-D capacitance
extraction algorithm described above.  In particular, the program
FASTCAP, which can use both direct factorization and
multipole-accelerated techniques, has been developed and incorporated
into MIT's MEMCAD (Micro-Electrical-Mechanical Computer-Aided Design)
system \cite{maseeh90}.  The structures described below were created
with the solid modeling program in the MEMCAD system, PATRAN, or by
computer program, and all capacitance calculations were performed
using FASTCAP.  The multipole-accelerated algorithms in FASTCAP use,
by default, second-order multipole expansions and a GMRES convergence
tolerance (see Algorithm 1) of $ 0.01 $.

%
% sphere and cube figures
%

%%&^^
\newcommand{\spherecap}{The sphere1 discretization of the unit sphere.}

\fig{\figuredir/sphere2.ps}{sphere2}{\spherecap}{215}{htbp}{104}{-12}{0.397059}{0.397059}


%
% sphere and cube figures
%

%%&^^
\newcommand{\cubencap}{The cube1 discretization of the unit cube.}

\fig{\figuredir/cuben5.ps}{cuben5}{\cubencap}{225}{htbp}{104}{-12}{0.397059}{0.397059}


To demonstrate absolute accuracy, the FASTCAP program was used to
compute the capacitance of a unit sphere, discretized as in
Figure~\ref{sphere2}, and a unit cube, discretized as in
Figure~\ref{cuben5}.  In Table~\ref{simpres}, the capacitances
computed using the PAMA algorithm are compared with the capacitances
computed using direct factorization of $ P $ in (\ref{eq:qtop}) (Direct),
and with analytic results for the unit sphere and with reference results
for the unit cube.  As can be seen from the table, the results using
the PAMA algorithm are easily within one percent of the analytic or
reference results.

\begin{table}
\begin{center}
\begin{tabular}{|l|c|c|}
\hline
Method & \multicolumn{2}{c|}{Problem} \\ \hline
       & Sphere1 & Cube1 \\ 
       & 768 panels & 150 panels \\ \hline
Direct & 110.6   & 73.26  \\ \hline 
PAMA  &  110.5   & 73.28  \\ \hline
Other &	$111$\dag &  $73.5$,$73.4$\\ 
 & & \\ \hline
\end{tabular}
\caption{Capacitance values (in pF) illustrating FASTCAP's accuracy. 
\dag By analytic calculation. 
\ddag From \protect\cite{ruehli73} \protect\cite{jawson77}.}
\label{simpres}
\end{center}
\end{table}

%%&^^
\newcommand{\weavecap}{The $2\times 2$ woven bus problem: bars have 
1m$\times $1m cross sections.  The three discretizations
are obtained by replacing each square face in (a) with the
corresponding set of panels in (b).}

\begin{figure}
\setlength{\unitlength}{1.0in}
\begin{picture}(6.5,3.2)
\put(0,0.2){
\special{psfile=\figuredir/weave2.ps,hoffset=-12,voffset=-12,hscale=0.397059,vscale=0.397059}
}
\put(3.45,0.2){
\special{psfile=\figuredir/weavediscr.ps,hoffset=-9,voffset=-9,hscale=0.298343,vscale=0.298343}
}
\thicklines
\put(4.7,2.6){Woven1}
\put(4.7,1.6){Woven2}
\put(4.7,0.6){Woven3}
\put(2.0,0){(a)}
\put(4.5,0){(b)}
\put(3.27,1.45){\vector(-1,0){0.4}}
\put(3.27,0.65){\line(0,1){2.1}}
\put(3.27,1.7){\line(1,0){0.2}}
\put(3.27,2.75){\line(1,0){0.2}}
\put(3.27,0.65){\line(1,0){0.2}}
\put(0.2,3.15){4}
\put(0.8,3.45){3}
\put(0.2,1.5){2}
\put(0.8,0.4){1}
\end{picture}
\caption{\weavecap}
\label{weave}
\end{figure}


The PAMA algorithm is nearly as accurate as the direct factorization
method even on more complex problems, such as the $ 2 \times 2$ woven bus
structure in Figure~\ref{weave}.  The capacitances computed using the
two methods are compared in Table~\ref{weavres}, using coarse, medium,
and fine discretizations of the woven bus structure, also shown in
Figure~\ref{weave}.  Note that the coupling capacitance $C_{12}$
between conductors one and two, which is forty-times smaller than the
self-capacitance $ C_{11} $, is computed nearly as accurately with the PAMA
algorithm as with direct factorization.

\begin{table}
\begin{center}
\begin{tabular}{|l|c|c|c|c|c|c|}
\hline
Method & \multicolumn{6}{c|}{Problem} \\ \hline
 & \multicolumn{2}{c|}{Woven1} 
 & \multicolumn{2}{c|}{Woven2} 
 & \multicolumn{2}{c|}{Woven3} \\ 
 & \multicolumn{2}{c|}{1584 Panels} 
 & \multicolumn{2}{c|}{2816 Panels} 
 & \multicolumn{2}{c|}{4400 Panels} \\ \cline{2-7}
       & $ C_{11} $ & $ C_{12} $ & $ C_{11} $ & $ C_{12} $ 
       & $C_{11} $ & $ C_{12} $ \\ \hline
%Direct & 251.6  &  $-6.352$  & 253.2 & $-6.447$ &  253.7 &  $-6.467$  \\\hline
Direct & 251.6  &  $-6.353$  & 253.2 & $-6.446$ &  253.7 &   $-6.467$  \\\hline
%PAMA  & 251.8  &  $-6.247$  & 253.4 & $-6.334$ & 253.9  &  $-6.377$  \\\hline
PAMA  & 251.8  &  $-6.246$  & 253.3 & $-6.334$ & 253.9  &  $-6.377$  \\\hline
\end{tabular}
\caption{Capacitance values (in pF) 
illustrating FASTCAP's accuracy for the complicated
geometry of Figure~\protect\ref{weave}.}
\label{weavres}
\end{center}
\end{table}

%
% the CPU time as a function of problem size graph
%

%%&^^
\newcommand{\growthcap}{CPU time as a function of problem size times
number of conductors, $mn$, for the PAMA algorithm applied to woven
buses of various sizes. The dashed line is the least-squares best fit
to the illustrated data points.}

\fig{\figuredir/growth.ps}{growth}{\growthcap}{333}{htbp}{-67}{-388}{0.955752}{0.955752}


The computational cost of using the FASTCAP program is roughly
proportional to the product of the number of conductors, $ m $, and
the number of panels $ n $.  This is experimentally verified using a
parameterized version of the woven bus structure in
Figure~\ref{weave}, that is, the structure is extented to make a $ 3
\times 3 $ bus, a $ 4 \times 4 $ bus and a $ 5 \times 5 $ bus, with
surfaces discretized as depicted in the $Woven1 $ example of
Figure~\ref{weave}. In Figure~\ref{growth}, the CPU times for
computing the capacitances of the woven bus structures are plotted as
a function of $ mn $, and as the graph clearly demonstrates the
computation time does grow linearly.

%%&^^
\newcommand{\examcap}{Two signal lines passing through conducting planes; via 
centers are 2mm appart.}

\begin{figure}
\setlength{\unitlength}{1.0in}
\begin{picture}(6.5,2.5)
\put(1.4,0){
\special{psfile=\figuredir/exam.ps,hoffset=-15,voffset=-15,hscale=0.489706,vscale=0.489706}
}
\put(1.25,1.8){2}
\put(1.3,1.2){1}
\put(1.9,0.6){3}
\put(2.6,0.15){4}
\end{picture}
\caption{\examcap}
\label{exam}
\end{figure}


%%&^^
\newcommand{\sqdcap}{A schematic illustration of the diaphragm problem. The 
two plates are $0.02\mu $m appart at the center.}

\begin{figure}
\setlength{\unitlength}{1.0in}
\begin{picture}(6.5,2.5)
\put(1.75,0){
\special{psfile=\figuredir/sqd.ps,hoffset=-12,voffset=-12,hscale=0.397059,vscale=0.397059}
}
\put(1.9,0.7){$1000\mu $m}
\put(3.9,0.6){$1000\mu $m}
\put(4.55,1.2){$1\mu $m}
\end{picture}
\caption{\sqdcap}
\label{sqd}
\end{figure}


To demonstrate the effectiveness of various aspects of the PAMA
algorithm on a range of problems, in Table~\ref{main} the CPU times
required to compute the capacitances of six different examples using
four different methods are given.  The examples Cube2 and Sphere2
are finer discretizations of the unit cube and sphere in
Figures~\ref{sphere2} and \ref{cuben5}; the examples 2 $ \times $ 2
Woven Bus and 5 $ \times $ 5  Woven Bus are described
above; the example Via, shown in Figure~\ref{exam}, models a
pair of connections between integrated circuit pins and a
chip-carrier; and the example Diaphragm, shown in
Figure~\ref{sqd}, is a model for a microsensor \cite{johnso91}.

From Table~\ref{main}, it can be seen that using the adaptive
multipole algorithm (AMA) typically reduces the computation time by a factor
of two over using the multipole algorithm (MA) alone, and that using the
preconditioner can reduce the computation time as much as a factor of
three.  Also, note that the PAMA algorithm is more than two
orders of magnitude faster than direct methods for the larger
problems.

%
% the convergence plot for the sqd problem
%

%%&^^
\newcommand{\sqdconvcap}{The GMRES residual norms for the linear system
solution corresponding to the Diaphragm problem (Figure~\protect\ref{sqd})
with the top conductor at unit potential. As is evident here, the AMA
method generally makes the iterate calculation more efficient, while
the PAMA algorithm leads to fewer iterations.}

\fig{\figuredir/sqdconv.ps}{sqdconv}{\sqdconvcap}{333}{htbp}{-67}{-388}{0.955752}{0.955752}


The reduction in execution time afforded by the adaptive algorithm
is due to increased efficiency in calculating each iterative loop iteration,
rather than in a reduction the total number of iterations.
The convergence 
plot in Figure~\ref{sqdconv} correspond to
one of the linear system solutions associated with the Diaphragm problem and
is representative of the general case.
As is evident in the figure, the AMA algorithm generally computes
nearly identical iterates to those calculated by the MA method but is faster,
as reported in Table~\ref{main}. In contrast, the preconditioner
leads to shorter computation time by 
reducing the total number of iterations rather than making each
iterate calculation less costly.  In fact one PAMA iterate requires
more computation than an AMA iterate, but the difference is more than
offset by the reduction in total number of 
iterations, as is certainly the case in Figure~\ref{sqdconv}. 

\begin{table}
\footnotesize
\begin{center}
\begin{tabular}{|l|c|c|c|c|c|c|c|}
\hline
Method & Cube2 & Sphere2 & 2x2 Woven Bus & Via & Diaphragm & 5x5 Woven Bus \\ 
       & 294 Panels & 1200 Panels & 4400 Panels & 6185 Panels & 7488 Panels & 9630 Panels\\
\hline
%Direct & 0.08 & 4.4 & 185 & (490) & (890) & (1920)\\ \hline
%MA     & 0.11 & 1.0 & 6.5 & 11    & 11    & 40\\ \hline
%AMA    & 0.03 & 0.4 & 2.7 & 3.8   & 9.0   & 19\\ \hline
%PAMA   & 0.03 & 0.4 & 1.9 & 3.1   & 3.0   & 12\\ \hline
Direct & 0.11 & 3.2 & 185 & (490) & (890) & (1920)\\ \hline
MA     & 0.06 & 0.3 & 6.0 & 11    & 8.7    & 42\\ \hline
AMA    & 0.05 & 0.2 & 3.3 & 4.7   & 5.9   & 23\\ \hline
PAMA   & 0.05 & 0.2 & 2.3 & 3.2   & 1.3   & 11\\ \hline
\end{tabular}
\caption{CPU times in minutes on an IBM RS6000/540, times in 
parentheses are extrapolated.}
\label{main}
\end{center}
\end{table}














