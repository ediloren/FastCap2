\section{Introduction}

In the design of high performance integrated circuits and integrated
circuit packaging, there are many cases where accurate estimates of
the capacitances of complicated three-dimensional structures are
important for determining final circuit speeds and functionality.
Algorithms using method of moments \cite{harrin68}
or weighted-residuals \cite{cranda56,collat66} 
based discretizations of
integral equation formulations,
also known as boundary-element methods \cite{brebbi84}, are commonly used to
compute these capacitances, but such approaches generate dense
matrix problems which are computationally expensive to solve,
and this limits the complexity of problems which can be analyzed.

In \cite{nabors91}, a fast algorithm for computing the capacitance of
three-dimensional structures of rectangular conductors in a homogenous
dielectric is presented.  The method solves the discretized capacitance
problem using an iterative technique with iterates computed by
a hierarchical multipole algorithm \cite{greeng87,greeng88}. This
general strategy was first suggested in \cite{rokhli85}.
The computation time for the algorithm was shown to grow nearly as $ mn$, 
where $ n $ is the
number of panels used to discretize the conductor surfaces, and $ m $
is the number of conductors.  In this paper, we describe several
improvements to that algorithm and present computational results on a
variety of examples to demonstrate that the new method is accurate and
can be as much as two orders of magnitude faster than standard
direct factorization approaches.

The outline of the paper is as follows.  The boundary-element
formulation and a standard iterative algorithm for solving the
generated matrix problem are briefly reviewed in Section~\ref{form}. A
simplified version of the hierarchical multipole algorithm is described
in Section~\ref{mutapp}, and our new adaptive multipole algorithm tuned to the
boundary-element formulation is given in Section~\ref{adfamu}.  A new
preconditioning strategy for accelerating the iterative algorithm,
based on the idea of screening, is presented in Section~\ref{precon}.
Experimental results using our program FASTCAP to analyze a wide
variety of structures, made possible by a link to the M.I.T.\
Micro-Electro-Mechanical Computer Aided Design (MEMCAD) system \cite{maseeh90},
are presented in Section~\ref{expres}.  
Finally, conclusions and acknowledgements
are given in Section~\ref{conclu}.

\section{Problem Formulation}
\label{form}

Capacitance extraction is made tractable by assuming the problem
contains conductors embedded in a homogenous dielectric medium, though
the techniques described below can be extended to the
piecewise-constant dielectric case using the approach in~\cite{rao84}.
The capacitance of an $m$-conductor geometry can then
be summarized by an $m\times m$ symmetric matrix $C$, where the $j$-th
column of $C$ has a positive entry $C_{jj}$, representing the
self-capacitance of conductor $j$, and negative  entries
$C_{ij}$, representing coupling between conductors $j$ and $i$, $i=1$,
2,\ldots, $m$, $i\neq j$.  To determine the $ j $-th column of the
capacitance matrix, one need only solve for the surface charges on
each conductor produced by raising conductor $ j $ to one volt while
grounding the rest.  Then $C_{ij}$ is numerically equal to
the charge on conductor $i$,
$i=1$, 2,\ldots, $m$.  Repeating this procedure $m$ times gives the
$m$ columns of $C$.

These $m$ potential problems can be solved using an equivalent
free-space formulation in which the conductor-dielectric interfaces
are replaced by a charge layer of density $\sigma$
\cite{ruehli73,rao84}.  Assuming a homogenous dielectric, the
charge layer in the free-space problem will be the induced charge in
the original problem if $ \sigma $ satisfies the integral equation
\begin{equation}
\psi(x)=\int_{  sur\!f\!aces } \sigma (x')\frac{1}{4\pi\epsilon_0\| x - x' \| } da',\rule{0.25in}{0in}x\in sur\!f\!aces.
\label{eq:integral}
\end{equation}
where $ \psi(x) $ is the known conductor surface potential, 
$da'$ is the incremental conductor surface area,
$ x$, $x' \in \mbox{\bf R}^3 $, 
and $ \| x \| $ is the usual Euclidean length of $x $ given by
$ \sqrt{x_1^2 + x_2^2 + x_3^2} $. 

A standard approach to numerically solving (\ref{eq:integral}) for $
\sigma $ is to use a piece-wise constant collocation scheme. That is,
the conductor surfaces are broken into $ n $ small panels or tiles, and
it is assumed that on each panel $ i $, a charge, $ q_i$, is uniformly
distributed.  Then for each panel, an equation is written which
relates the known potential at the center of that $ i$-th panel,
denoted $\overline{p}_i $, to the sum of the contributions to that
potential from the $ n $ charge distributions on all $ n $ panels
\cite{rao84}.  The result is a dense linear system,
\begin{equation}
P q = \overline{p}
\label{eq:qtop}
\end{equation}
where $ P \in \mbox{\bf R}^{n\times n}$, $q$ is the
vector of panel charges, $\overline{p} \in \mbox{\bf R}^n$ 
is the vector of known panel potentials, and
\begin{equation}
P_{ij} = \frac{1}{a_j} 
\int_{panel_j} \frac{1}{ 4\pi\epsilon_0\|  x_i-x' \| }\: da',
\label{eq:pcoeff}
\end{equation}
where $ x_i $ is the center of the $ i$-th panel and $ a_j $ is the
area of the $ j$-th panel. Since the discretiztion uses point collocation,
in general $P_{ij} \neq P_{ji}$, that is $P$ is unsymmetric.

The dense linear system of (\ref{eq:qtop}) can be solved to compute
panel charges from a given set of panel potentials, and the
capacitances can be derived by summing the panel charges.  If Gaussian
elimination is used to solve
(\ref{eq:qtop}), the number of operations is order $ n^3 $.  Clearly,
this approach becomes computationally intractable if the number of
panels exceeds several hundred.  Instead, consider solving the linear 
system (\ref{eq:qtop}) using a conjugate-residual style iterative
method like GMRES \cite{saad86}.  Such methods have the form given
below:

\begin{singlespace}
\begin{quote}
\begin{tabbing}
{\bf Algorithm 1:} GMRES algorithm for solving (\ref{eq:qtop})\\
Make an initial guess to the solution, $q^0$.\\
Set $k = 0$.\\
T \= \kill
\>T \= \kill
\>\>T \= \kill
\>\>\>T \= \kill
\>\>\>\>T \= \kill
\>\>\>\>\>T \= \kill
{\bf do} \{\=\\
\>Compute the residual, $r^k=\overline{p}-Pq^k$.\\
\>{\bf if}  $\|r\|<tol$, return $q^k$ as the solution.\\
\>{\bf else} \{\=\\
\>\>{\bf Choose} $ \alpha $'s and $ \beta $ in \\
\>\>\>\>\>\> $q^{k+1} = \sum_{j=0}^k \alpha_j q^j + \beta r^k$\\
\>\>to minimize $ \| r^{k+1} \| $.\\
\>\>Set $k = k+1$.\\
\>\}\\
\}
\end{tabbing}
\end{quote}
\end{singlespace}

The dominant costs of Algorithm 1 are in calculating the $ n^2 $
entries of $ P $ using (\ref{eq:pcoeff}) before the iterations begin,
and performing $n^2$ operations to compute $ P q^k $ on each
iteration.  Described below is our adaptive hierarchical multipole
algorithm which, through the use of carefully applied approximations,
avoids forming most of $ P $ and reduces the cost of forming $ P q^k $
to order $ n $ operations.  
This does not necessarily imply that each
iteration of the GMRES algorithm can be computed with order $ n $
operations.  If the number of GMRES iterations required to achieve
convergence approaches $ n $, then to perform the minimization in each
GMRES iteration will require order $ n^2 $ operations.  This problem
is avoided through the use of a preconditioner, also described below,
which reduces the number of GMRES iterations required to achieve
convergence to well below $ n $ for large problems.



