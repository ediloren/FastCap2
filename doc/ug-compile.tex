%
% Description of output compile flag effects and compile procedure
%
\section{Compiling FastCap}
\label{comfas}

%\noindent{\sc Check particulars w/Miguel.}

A tar file containing the source files for {\tt fastcap},
{\tt busgen}, {\tt capgen}, {\tt cubgen} and this guide, as well as
all the neutral files it describes, may be obtained on tape
by sending a written
request to
\begin{quote}
Prof.\ Jacob White\\
Massachusetts Institute of Technology\\
Department of Electrical Engineering and Computer Science\\
Room 36-880\\
Cambridge, MA 02139 U.S.A.
\end{quote}
This address may also be used for  general correspondence regarding 
{\tt fastcap}, although electronic mail may be sent to
{\tt fastcap-bug@rle-vlsi.mit.edu}, 
for bug reports, and to {\tt fastcap@rle-vlsi.mit.edu}, for
questions or comments, if it is more convenient.

The tar file has the form
\begin{quote}
\begin{verbatim}
fastcap-2.0-25May92.tar.Z
\end{verbatim}
\end{quote}
and yields a two level directory tree when untarred with the commands
\begin{quote}
\begin{verbatim}
uncompress fastcap-2.0-25May92.tar.Z
tar xvf fastcap-2.0-25May92.tar
\end{verbatim}
\end{quote}
The top-level directory is called {\tt fastcap} and has four subdirectories.
The directory {\tt fastcap/src} contains all the C source files; 
the \LaTeX\ 
files for this manual are found in {\tt fastcap/doc},  
and {\tt fastcap/examples} contains
the example files. Executables are stored in the {\tt fastcap/bin}
directory after compilation using the procedures described below.

\subsection{Default Compilation Procedure}

The default version of the executable {\tt fastcap} is 
compiled\footnote{For IBM AIX systems, it is recommended that
{\tt fastcap/src/Makefile} be reconfigured before compilation
using the procedure described in
Section~\ref{specom}.}
by changing to the top-level directory, {\tt fastcap}, and typing 
\begin{quote}
\begin{verbatim}
make all
\end{verbatim}
\end{quote}
to install the executables {\tt fastcap}, 
{\tt busgen}, {\tt capgen} and
{\tt cubegen} in the {\tt fastcap/bin} directory. Alternatively any of the
executables may be compiled alone. For example, ``{\tt make fastcap}''
produces only {\tt fastcap}.

\subsection{Special Compilation Procedures}
\label{specom}

If {\tt fastcap} is configured to print execution time information
by setting the {\tt TIMDAT} flag in {\tt src/mulGlobal.h} to {\tt ON}
(see Table~\ref{outpu}), an alternate compilation procedure is
needed to account for operating system differences.
To configure {\tt fastcap/src/Makefile}, change to the top-level
directory, {\tt fastcap}, and  type
\begin{quote}
\begin{verbatim}
config 4
\end{verbatim}
\end{quote}
for 4.2 or 4.3 BSD systems,
\begin{quote}
\begin{verbatim}
config 5
\end{verbatim}
\end{quote}
for System V systems or
\begin{quote}
\begin{verbatim}
config aix
\end{verbatim}
\end{quote}
for IBM AIX systems. The AIX configuration is recommended for AIX systems
even when {\tt fastcap} is not configured to print execution time
information. Also certain DEC compilers run out of space during the
compile.  For those machines use
\begin{quote}
\begin{verbatim}
config dec
\end{verbatim}
\end{quote}
The default configuration may be restored by typing
\begin{quote}
\begin{verbatim}
config
\end{verbatim}
\end{quote}
With {\tt fastcap/src/Makefile} configured, executables are
compiled as in the default case.

\subsection{Configuration Flags}
\label{confla}

The file {\tt src/mulGlobal.h} defines 
flags that configure the discretization,
multipole and output routines used in the executable {\tt fastcap}. 
Table~\ref{outpu} describes the   output configuration flags in more detail.
\begin{table}
\begin{center}
\begin{tabular}{lllp{3.5in}}\hline
\multicolumn{1}{c}{Flag}& \multicolumn{1}{c}{Default} 
&\multicolumn{2}{c}{Function}\\\hline
{\tt MKSDAT} & {\tt ON} & {\tt ON} & Prints the symmetrized capacitance matrix.\\
&& {\tt OFF} & Prints nothing.\\
{\tt CMDDAT} & {\tt ON} & {\tt ON} & Prints the command line arguments.\\
&& {\tt OFF} & Prints nothing.\\
{\tt RAWDAT} & {\tt OFF} & {\tt ON} & Prints the unsymmetrized capacitance matix.\\
&& {\tt OFF} & Prints nothing.\\
{\tt ITRDAT} & {\tt OFF} & {\tt ON} & Prints the residual norm after each iteration.\\
&& {\tt OFF} & Prints the iteration number after each iteration.\\
{\tt TIMDAT} & {\tt OFF} & {\tt ON} & Prints a summary of CPU time and memory usage. Times are
only reported if {\tt fastcap} is compiled using the procedure of 
Section~\ref{specom}.\\
&& {\tt OFF} & Prints nothing.\\
{\tt CFGDAT} & {\tt OFF} & {\tt ON} & Prints core configuration flags.\\
&& {\tt OFF} & Prints nothing.\\
{\tt MULDAT} & {\tt OFF} & {\tt ON} & Prints brief multipole setup information.\\
&& {\tt OFF} & Prints nothing.\\
{\tt DISSYN} & {\tt OFF} & {\tt ON} & Prints summary of cube involvement by partitioning level.\\
&& {\tt OFF} & Prints nothing.\\
{\tt DMTCNT} & {\tt OFF} & {\tt ON} & Prints the number of multipole transformation matrices for all possible cube pairings.\\
&& {\tt OFF} & Prints nothing.\\
{\tt DISSRF} & {\tt ON} & {\tt ON} & Prints surface file information.\\
&& {\tt OFF} & Prints nothing.\\\hline
\end{tabular}
\end{center}
\caption{{\tt fastcap} output configuration compile flags (defined in {\tt fastcap/src/mulGlobal.h}).}
\label{outpu}
\end{table}

\subsection{Producing this Guide and Related Documents}

In the top-level directory, {\tt fastcap}, type
\begin{quote}
\begin{verbatim}
make manual
\end{verbatim}
\end{quote}
to produce three {\tt .dvi} files {\tt ug.dvi}, 
{\tt tcad.dvi} and {\tt mtt.dvi} in the directory {\tt fastcap/doc}.
The file {\tt ug.dvi} generates
the user's guide, while {\tt tcad.dvi} and {\tt mtt.dvi}
produce reprints of two articles describing the algorithm
used by {\tt fastcap}.
